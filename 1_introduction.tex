
\section{\Large \textbf{Significance to the real world}}
{ Analyzing layoffs in the top 20 affected states of the US is significant in order to understand how the economic dynamics are different from those of social and policy ones. It helps in providing insights into how the economic health of regions is impacted and it also indicates the area of growth or decline. By identifying such kinds of industries we are able to deal with the most significant challenge that is present. The policymakers will get a streamlined set of retraining programs or targeted support. In addition to this, understanding the social impact of layoffs is very important as it also includes the employment rates and the potential migration patterns which indeed helps allocating the resources which are effective enough to support the affected individuals and communities. This analysis informs strategic planning for businesses that anticipate the industry trends and also identify the opportunities that are essential for growth or diversification. Altogether, examining the layoffs in these 20 states offers valuable real-world data that not only informs the decisions of policymakers but also investors, job seekers, and businesses, who have more informed and effective responses to the challenges with respect to the economy and growth.
}
\section{\Large \textbf{Vital lessons learned from the Project}}
{In this whole project, we have learned about many valuable lessons which include industry vulnerabilities, policy effectiveness, and also the economic resilience which can be regional. This project has helped us to highlight certain industries that are more prone to layoffs due to multiple factors that have come up like automation and globalization. The study also highlights variations in regional economic resilience, providing information on things like worker adaptation and community responses. By gaining an understanding of the dynamics of corporate practices during layoffs and long-term employment trends, policymakers, businesses, and communities can better execute targeted interventions, promote economic resilience, and support workers in affected states.
}

\section{\Large\textbf{Innovation}}
{In order to structure the lessons that we have learned from this project innovation is an essential part of it to understand the effects of layoffs and address the various vulnerabilities in the industry, Innovative strategies must be developed to handle such problems. One of the examples of such strategies includes developing new technologies, business models, or   training the workers. In addition to that there are various different approaches which are being researched to an enhance their effectiveness to support such workers  and fostering economic residence such as Novel policy. The study also examines how companies use the technology in the creative way and calculate the moves to get around the challenges and capture that chances as the employment landscape changes. At the time of the layoffs, stakeholders can effectively manage any employment difficulties and Continue to move forward in long term for economic growth by embracing the innovation in corporate tactics, policy making, and Community responses. 
}

\section{\Large\textbf{Teamwork}}
{In this project teamwork is an integral part to achieve our goals. each member of the team contributes that own specialized skills and knowledge, fostering a collaborative environment where diverse perspectives are valued. through our effective communication and coordination, Rihanna's the collective expertise to conduct the rigorous data analysis, did I have meaningful insights, and develop various actionable strategies to address employment challenges in the affected States.
}

\subsection{Roles} 
{\textbf{Data Extraction -} Pranav and Dhruv\\
\textbf{Data Cleaning using Python}  - Pranav and Dhruv\\
\textbf{Data Modeling (ER diagrams)} - Aishwarya, Sheetal, Lincy, Pranav, Dhruv\\
\textbf{Data Warehouse -} Aishwarya, Lincy\\
\textbf{Design in MySQL -} Aishwarya, Sheetal\\
\textbf{Data Analysis-} Aishwarya, Sheetal\\
\textbf{Data Visualization - }Aishwarya, Sheetal\\
\textbf{Generation of Analysis Report  -} Lincy, Sheetal\\
\textbf{Report Writing in Latex -} Lincy, Sheetal, Aishwarya, Pranav.}

\section{\Large\textbf{Technical difficulty}}
\begin{itemize}
    \item {The primary issue that we encountered was that the data set was extremely cluttered and messy and very disorganized. It was very difficult to obtain up to date and trustworthy  layoff data from the states that were impacted, especially because of the possible delays in reporting in the  WARN  act . Now to make sure that we have a complete and current data for analysis we are  finding alternate sources such as respectable news organizations and research institutions that frequently release State specific layoff data.
\end{itemize}
\begin{itemize}
    \item It can be difficult to integrate the data that is mixed from many different sources and formats. Due to which before analyzing be standardize the data formats as a part of a solution. To guarantee the correctness and the consistency across the data sets, which means this may require making manual tweaks or using data cleaning technologies, allowing for smooth integration for Reliable analysis.
\end{itemize}
\begin{itemize}
    \item Sorting through massive volumes of data of employment laws from several States can be very difficult, especially given the different economic environments . To address this,  we are using different tactics such as your categorizing States based on their economic sectors, utilizing various visualization tools for more detail insights, and doing comparison analysis to identify different patterns and  and to acquire  a  thorough grasp Off a bigger economic picture. These various methodologies allows us to successfully navigate the Complexity of the data and extract important insights to inform our study.
\end{itemize}

\section{\Large\textbf{Pair Programming}}
{In our project, we embraced pair programming to address the issue of dealing with a messy dataset. One team member assumed the role of the "driver," responsible for hands-on coding tasks like data cleaning and manipulation. Meanwhile, the other team member served as the "navigator," actively reviewing the code, suggesting improvements, and ensuring alignment with project objectives. This dynamic collaboration allowed us to efficiently tackle challenges, enhance code quality, and foster knowledge exchange within the team.
}


\section{\Large\textbf{Scrum Practice}}
{For this project, we've implemented Agile methodologies, specifically Scrum, utilizing one-week sprints to manage our project effectively. We've established regular sprint planning, weekly stand-up meetings, and sprint reviews to ensure continuous progress and adaptability to changing requirements. Our team maintains transparency and accountability through detailed meeting minutes, task boards, and other artifacts, which are regularly updated and submitted on Canvas for documentation and review. 
}\\
\vspace{2cm}
\includegraphics[width=0.9\textwidth]{root/ss1.png}~\\
\includegraphics[width=0.9\textwidth]{root/ss2.png}~\\
{We did make it a regular practice to jot down the minutes of meeting and did share it to all the team members.One of the example is here:}
\href{https://docs.google.com/document/d/1qKK7A6k3jXr8TkXPlfGVerHoZhj3we9W/edit?usp=drive_link&ouid=117701647846204913682&rtpof=true&sd=true
}{MOM of the DBA project meeting.}


\section{\Large\textbf{Usage of Grammarly}}
{Using Grammarly in our project report was one of the best decisions that we took as it really helped us in making our content very much polished and error-free. The suggestions  that 	Grammarly gave have helped us in correcting all of our spelling mistakes and grammar mistakes. We personally do feel that there is a huge improvement in the sentence structure that we are currently forming and it is maintaining a consistent writing style which is quite crucial in writing reports. There is a significant enhancement in the overall clarity and professionalism of our report which will definitely make it easier for the readers to understand our findings and recommendations. This assistance has saved a lot of time and effort in terms of proofreading and has allowed us to focus more on the content and analysis that we have performed for our project.
}\\
\vspace{2cm}
\includegraphics[width=0.9\textwidth]{root/ss3.png}~\\
\vspace{2cm}

\section{\Large\textbf{Performed substantial analysis using database techniques}}
{We've developed two pipelines for loading various CSV files into MySQL Workbench. In the first approach, we utilize Python to ingest the CSV files, conduct cleaning and transformation operations within Python, and subsequently establish a MySQL connection via Python to load the transformed data into MySQL Workbench tables. 
Alternatively, in the second pipeline, we leverage Apache NiFi as our ETL tool to read the data, perform transformations into JSON format, and then convert the JSON data into SQL. This SQL data is then loaded into MySQL Workbench tables.
}\\
\vspace{2cm}
\includegraphics[width=0.9\textwidth]{root/ss4.png}~\\\vspace*{-1cm}
{For the ETL tool, we have used Apache NiFi.}
\vspace{2cm}\\
\vspace{2cm}
\includegraphics[width=0.9\textwidth]{root/ss5.png}~\\
{Once the data is in MySQL Workbench, we conduct various analyses, a few of which are listed below with the screenshots:}
\vspace{2cm}
\begin{itemize}
    \item \textbf{Obtaining the total number of layoffs by states.}
\end{itemize}

\vspace{2cm}
\includegraphics[width=1.0\textwidth]{root/ss20.png}~\\\vspace*{-1cm}
\textbf{{OUTPUT}}\\

\includegraphics[width=0.6\textwidth]{root/ss6.png}~\\\vspace*{-1cm}

\begin{itemize}
    \item \textbf{Calculating the total number of layoffs by Closure/Layoff Statuses.}

\end{itemize}

\includegraphics[width=1.0\textwidth]{root/ss7.png}~\\\vspace*{-1cm}
\textbf{{OUTPUT}}\\

\includegraphics[width=0.6\textwidth]{root/ss8.png}~\\\vspace*{-1cm}

\begin{itemize}
    \item \textbf{Aggregating the total number of layoffs by city.}

\end{itemize}

\includegraphics[width=1.0\textwidth]{root/ss9.png}~\\\vspace*{-1cm}
\textbf{{OUTPUT}}\\

\includegraphics[width=0.5\textwidth]{root/ss10.png}~\\
{These analyses allow us to gain insights into the distribution of layoffs across different geographical regions and closure statuses, enabling informed decision-making and strategic planning.}


\section{\Large\textbf{Used a new database or data warehouse tool not covered in the HW or class}}
{In addition to the pipelines mentioned above, we've established a data warehouse database in\textbf{ Snowflake} and configured tables within it. So in this, we have loaded the transformed data into this Snowflake database, which enables the reporting capabilities. The screenshots below show the visual confirmation of our setup and our data-loading processes.
}\\
\includegraphics[width=1.0\textwidth]{root/ss11.png}~\\[1cm]
\includegraphics[width=1.0\textwidth]{root/ss12.png}~\\

\section{\Large\textbf{Used appropriate data modeling techniques
}}
{In this project, we have used to data modeling techniques to understand the different patterns of layoffs across the most affected states in the United States.}\\

\begin{enumerate}
    \item \textbf{Entity relationship modeling} we use this technique to identify and Define the entities such as the state industry companies and the relationships such as layoff affecting employee rates Etc within the given data set by using this ER relationship diagrams. It helps us to visualize the relationships and to make it easier to understand the complex data structures.\\[1cm]
\textbf{{ER Diagram}}\\

\includegraphics[width=1.0\textwidth]{root/ss13.png}~\\


\item \textbf{Normalization} We have used normalization and we applied it to ensure that the database is organized efficiently which reduces the redundancy issues and improves the data integrity. The normalization techniques that we have applied are First Normal Form (1NF), Second Normal Form (2NF), and Third Normal Form (3NF), to ensure data integrity and optimize database performance.\\
\includegraphics[width=1.0\textwidth]{root/ss14.png}~\\

\item \textbf{Dimensional modeling} by using this technique we organize the data into different dimensions and facts such as the number of layoffs unemployment rate Etc this basically creates a data warehouse which is optimized for analytical queries which enables us to analyze the layoffs from different perspectives easily. We've established a data warehouse database in \textbf{Snowflake} and configured tables within it. 
\\

\item \textbf{Data Modelling Tool:} We have leveraged the usage of the MySQL database to perform various data modeling activities.\\

\end{enumerate}

\section{\Large\textbf{Used ETL tool}}
{We have used Apache NiFi as the ETL tool which has played a very crucial role in our project workflow. We have used this to read the data, perform transformations into JSON format, and then convert the JSON data into SQL. This SQL data is then loaded into MySQL Workbench tables. This helped in standardizing the data which ensured the data was consistent and accurate.}\\

{We have also utilized Python as an ETL Tool to ingest the CSV files, post which we also performed the cleaning and transformation operations within Python. Further, we established a MySQL connection via Python in order to load the transformed data into MySQL Workbench tables.}\\
{The screenshot attached below is the one that is used to create a pipeline in the ETL tool which is Apache NiFi.}\\

\includegraphics[width=1.0\textwidth]{root/ss15.png}~\\

\section{\Large\textbf{Demonstrated how Analytics support business decisions}}
{Through this analysis, we aim to understand the analytical role to make informed business decisions. We use data visualization techniques to understand and gain different insights and through our analysis, it showed the how businesses can utilize our analytics to identify areas of concern and understand future trends and strategize  and plan the impact of the layoff band having an understanding beforehand organizations can make more informed decisions regarding the management planning and any resource allocation a project highlights the  potential in empowering businesses to navigate through challenging economic times and enhance their decision making for long term success.}\\

\includegraphics[width=1.0\textwidth]{root/ss16.png}~\\

{The graph breaks down layoffs into different categories, such as closures (plant, facility), temporary and permanent layoffs, and mass layoffs. By analyzing the distribution of these categories across states, you can identify which types of layoffs are most prevalent in different regions.}

\includegraphics[width=1.0\textwidth]{root/ss17.png}~\\
{This shows the top 25 companies with the highest number of layoffs”. It lists companies like Yellow Corporation, USI service groups, US Airways, United Airlines, Macy’s, and Bank of America as the top listed companies with highest layoffs.}

\section{\Large\textbf{Used RDBMS}}
{We used  RDBMS - \textbf{MYSQL} for this particular project which played a crucial role in the structuring of the data which would then be defined into tables and relationships. The  SQL queries also helped us in analyzing the data based on the different states Industries and other dimensions it also helped us in understanding and performing complex joins and aggregations for detailed analysis.}\\

\section{\Large\textbf{Used Data warehouse }}
{We've created a data warehouse in \textbf{Snowflake} and configured tables within it. So in this, we have loaded the transformed data into this Snowflake database. There is continuous improvement through the usage of this and the performance monitoring and decision making is also enhanced.}\\

\section{\Large\textbf{Includes DB Connectivity}}
{We utilize Python to ingest the CSV files, conduct cleaning and transformation operations within Python, and subsequently \textbf{establish a MySQL connection via Python} to load the transformed data into MySQL Workbench tables. 
The attached screenshot below shows how the connection is established.}\\
\includegraphics[width=0.8\textwidth]{root/ss18.png}~\\


\section{\Large\textbf{Used NOSQL}}
{We have leveraged the usage of MongoDB for the NoSQL part as NoSQL databases are very important, particularly for managing a different data types which gives us scalability. NoSQL database provides us flexibility in storing such data. We use NoSQL in our project which helps us understand the context of the layoffs in the US. We have performed aggregations in the MongoDB compass to perform essential analysis.}\\[1cm]
\includegraphics[width=0.8\textwidth]{root/ss19.png}~\\